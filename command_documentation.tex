\section{Commands}

\subsection{login}

\begin{description}
\item {\it Description:}\\
Login of the client/user onto the server. This command must be called before all others. This command will assure, that client and server speak the same version of the protocol.
\item {\it Parmameters:}
\begin{itemize}
\item {\tt protocol\_version}: version of the protocol, the client expects
\item {\tt client\_version}: name/version string of the client
\item {\tt client\_uniqid}: some uniq id of the client
\end{itemize}
\item {\it Result:}\\
The reply contains 4 cells: protocol version, server version, client id for further requests, ip the connection came from
\end{description}

\subsection{req\_full}

\begin{description}
\item {\it Description:}\\
Returns a list of races and the drivers in this races.
\item {\it Parmameters:}
\begin{itemize}
\item {\tt client\_id}: 
\end{itemize}
\item {\it Result:}\\
The complete current racelist. Each line holds either a race or following the drivers of a race. Each line starts either with a cell containing R or D. Races consist of the following: 
				
			R, 
			server\_id, 
			ip, 
			joinport, 
			name, 
			info1, 
			info2, 
			comment, 
			isdedicatedserver, 
			ispassworded, 
			isbosspassworded, 
			isauthenticedserver, 
			allowedchassis, 
			allowedcarclasses, 
			allowsengineswapping, 
			modindent, 
			maxlatency, 
			bandwidth, 
			players,
			maxplayers, 
			trackdir, 
			racetype, 
			praclength, 
			sessionleft, 
			sessiontype,
			aiplayers,
			numraces,
			repeatcount,
			flags


			The drivers contain this data:

			D,
			firstname,
			lastname,
			class\_id,
			team\_id,
			mod\_id,
			nationality,
			helmet\_colour,
			qualifying\_time,
			race\_position,
			race\_laps,
			race\_notes
			
\end{description}

\subsection{host}

\begin{description}
\item {\it Description:}\\
Starts hosting of a race. The given informations are used to describe the race and will be displayed in the same order in the racelist.
\item {\it Parmameters:}
\begin{itemize}
\item {\tt client\_id}: 
\item {\tt joinport}: 
\item {\tt name}: 
\item {\tt info1}: 
\item {\tt info2}: 
\item {\tt comment}: 
\item {\tt isdedicatedserver}: 
\item {\tt ispassworded}: 
\item {\tt isbosspassworded}: 
\item {\tt isauthenticedserver}: 
\item {\tt allowedchassis}: 
\item {\tt allowedcarclasses}: 
\item {\tt allowsengineswapping}: 
\item {\tt modindent}: 
\item {\tt maxlatency}: 
\item {\tt bandwidth}: 
\item {\tt maxplayers}: 
\item {\tt trackdir}: 
\item {\tt racetype}: 
\item {\tt praclength}: 
\item {\tt aiplayers}: 
\item {\tt numraces}: 
\item {\tt repeatcount}: 
\item {\tt flags}: 
\item {\tt firstname}: 
\item {\tt lastname}: 
\item {\tt class\_id}: 
\item {\tt team\_id}: 
\item {\tt mod\_id}: 
\item {\tt nationality}: 
\item {\tt helmet\_colour}: 
\end{itemize}
\item {\it Result:}\\
A unique id for the server, that will be used to update the hosting and race informations and also by the clients to join/leave the race and the IP address the request came from
\end{description}

\subsection{join}

\begin{description}
\item {\it Description:}\\
The client with the given id joins the server with the given id. Several informations about the driver itself are also submited for the list of races and their drivers.
\item {\it Parmameters:}
\begin{itemize}
\item {\tt server\_id}: 
\item {\tt client\_id}: 
\item {\tt firstname}: 
\item {\tt lastname}: 
\item {\tt class\_id}: 
\item {\tt team\_id}: 
\item {\tt mod\_id}: 
\item {\tt nationality}: 
\item {\tt helmet\_colour}: 
\end{itemize}
\item {\it Result:}\\
Nothing.
\end{description}

\subsection{leave}

\begin{description}
\item {\it Description:}\\
Removes the client with the given id from the server with the given id.
\item {\it Parmameters:}
\begin{itemize}
\item {\tt server\_id}: 
\item {\tt client\_id}: 
\end{itemize}
\item {\it Result:}\\
Nothing.
\end{description}

\subsection{endhost}

\begin{description}
\item {\it Description:}\\
Stops the hosting of the race with the given id.
\item {\it Parmameters:}
\begin{itemize}
\item {\tt server\_id}: 
\item {\tt client\_id}: 
\end{itemize}
\item {\it Result:}\\
Nothing.
\end{description}

\subsection{report}

\begin{description}
\item {\it Description:}\\
Updates the informations of the given server.
\item {\it Parmameters:}
\begin{itemize}
\item {\tt server\_id}: 
\end{itemize}
\item {\it Result:}\\
Nothing.
\end{description}

\subsection{copyright}

\begin{description}
\item {\it Description:}\\
Returns a copyright notice about the protocol and the server.
\item {\it Parmameters:}
\begin{itemize}
\item None
\end{itemize}
\item {\it Result:}\\
String holding the text of the copyright notice.
\end{description}

\subsection{help}

\begin{description}
\item {\it Description:}\\
Returns a list of all implemented commands.
\item {\it Parmameters:}
\begin{itemize}
\item None
\end{itemize}
\item {\it Result:}\\
For each command a line starting with command, then followed by a line for each param and finally a line starting with result, explaining the data sent back to the client.
\end{description}

