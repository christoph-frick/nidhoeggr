\documentclass[a4paper,10pt]{article}

\newif\ifpdf
\ifx\pdfoutput\undefined
\pdffalse
\else
\pdfoutput=1
\pdftrue
\fi

\ifpdf
\usepackage{color}
\definecolor{black}{rgb}{0,0,0}
\usepackage[
	pdftex,
	colorlinks=true,
	urlcolor=black,
	filecolor=black,
	linkcolor=black,
	pdftitle={Description of the scary protocol},
	pdfauthor={Christoph Frick <rid@zefix.tv>},
	pagebackref,
	pdfpagemode=UseOutlines,
	bookmarksopen=true]{hyperref}
\pdfcompresslevel=3
\fi

\newcommand{\scary}{{\it scary}}
\newcommand{\rls}{{\it Race List Server}}
\newcommand{\rlc}{{\it Race List Client}}
\newcommand{\gplop}{{\it iGOR}}
\newcommand{\gplopdev}{{\it iGOR Development Group}}

\newcommand{\scaryport}{27233}


\begin{document}

\title{Description of the \scary\ protocol}
\author{Phil Flack, Christoph Frick}

\maketitle

\vspace{2in}

\abstract

The \scary protocol is used between any \rlc\ and any \rls\ as middleware for
sending requests to the server adding, updating, querying and deleting
informations. This document describes the protocol itself and the commands
used over it.\\

The may only \scary\ protocol be used by applications, that are allowed to use
it. This is valid both for \rlc's and \rls's. The use without the written
permission of the \gplopdev is strictly forbidden.\\

Copyright (c) 2002-2004 Phil Flack
          
Copyright (c) 2002-2004 Christoph Frick
          
Copyright (c) 2002-2004 \gplopdev

All rights reserved

\newpage


\section{Protocol}

The protocol itself using TCP/IP for connections between the \rlc\ (client) and
the \rls\ (server). The client connects to the server on the default port
(\scaryport). The client send the {\it request} to the server. The server then
forges a {\it reply} to the request and sends it back to the client. The
server then ends the connections. Only one request/reply per connection is
sent.\\

The data representing the request or the reply are wrapped in the following
fashion:\\

{\tt IIIICSSSS}\\

\begin{description}

\item {\tt IIII}\\
This is the ident string for this protocol. The ident are the chars: {\tt w196} \\

\item {\tt C}\\
This flags defines whether the data are gnuzip compressed data or clear text.
For compressed data a {\tt C} is used - and for clear text a {\tt T}

\item {\tt SSSS}\\
This is the size of the request, the other side has to expect to read then.
The number is an {\it unsigned long int} in {\it network byte order}.  The
size of the data transmitted is always the real size. If using compressed
transmission of the data, the size is the length of the data {\bf after} the
compression.\\

\end{description}

After the interpretation of the header, the given amount of bytes is read from
the socket. The data are encoded in ISO-8859-1.\\

To allow a common way to seprate the different fields in the data the
following scheme is used to establish a table like handling of the data. The
char {\tt $\backslash$002} is used to separate {\bf lines}. The char {\tt
$\backslash$001} is used to separate {\bf cells/columns}.\\

No string informations of the programming language, which is used for the
implementation, are sent to the other side (e.g. $\backslash$000 chars in
C).\\

Within the data the command is always the the first cell. After the command
all the parameters to the command are the following cells in this line. The
parameters are in a fixed order. Also there are serverside checks for each
parameter.\\

The reply of the server contains in the first line the status of the reply.
The first line contains two cells. The first is a number, that stands for a
group of errors. The second cell contains an message, that gives an accurate
description for the number. The following numers are used:\\

\begin{description}

\item 200 \\
The request is fine - expect the data -- if there is some additional reply --
to be found in the following rows. The reply depends on the command. The
replies are described in the next section.

\item 400 \\
There is an error with the request. See the description for details. Most
common are errors with the amount or the type of the submitet parameters.

\item 401 \\
The execution of this command needs a valid client id. Either the user is not
logged in or the user has been dropped due to inactivity for a longer period.

\item 404 \\
The resource used in the request is not (longer) known. This is most common,
when sending server ids and the server is already dropped from the list due to
longer inactivity.

\item 500 \\
There is an internal server error. This should not happen, so it would be good
to have here a usefull description about, what command resulted in this error
and if possible how to reproduce this error. Inform the developer of the
server about this issue or the administrator of the server and try to provide
him with as many informations as possible.

\item 501 \\
The command is not yet implemented. If your client application depends on this
command, get in contact with the developer or the administrator of the server.

\end{description}

% this part is generated from the server
\section{Commands}

\subsection{join}

\begin{description}
\item {\it Description:}\\
The client with the given id joins the server with the given id. Several informations about the driver itself are also submited for the list of races and their drivers.
\item {\it Parmameters:}
\begin{itemize}
\item server\_id
\item client\_id
\item firstname
\item lastname
\item class\_id
\item team\_id
\item mod\_id
\item nationality
\item helmet\_colour
\end{itemize}
\item {\it Result:}\\
Nothing.
\end{description}

\subsection{leave}

\begin{description}
\item {\it Description:}\\
Removes the client with the given id from the server with the given id.
\item {\it Parmameters:}
\begin{itemize}
\item server\_id
\item client\_id
\end{itemize}
\item {\it Result:}\\
Nothing.
\end{description}

\subsection{endhost}

\begin{description}
\item {\it Description:}\\
Stops the hosting of the race with the given id.
\item {\it Parmameters:}
\begin{itemize}
\item server\_id
\item client\_id
\end{itemize}
\item {\it Result:}\\
Nothing.
\end{description}

\subsection{report}

\begin{description}
\item {\it Description:}\\
Updates the informations of the given server.
\item {\it Parmameters:}
\begin{itemize}
\item server\_id
\end{itemize}
\item {\it Result:}\\
Nothing.
\end{description}

\subsection{copyright}

\begin{description}
\item {\it Description:}\\
Returns a copyright notice about the protocol and the server.
\item {\it Parmameters:}
\begin{itemize}
\item None
\end{itemize}
\item {\it Result:}\\
String holding the text of the copyright notice.
\end{description}

\subsection{help}

\begin{description}
\item {\it Description:}\\
Returns a list of all implemented commands.
\item {\it Parmameters:}
\begin{itemize}
\item None
\end{itemize}
\item {\it Result:}\\
For each command a line starting with command, then followed by a line for each param and finally a line starting with result, explaining the data sent back to the client.
\end{description}

\subsection{login}

\begin{description}
\item {\it Description:}\\
Login of the client/user onto the server. This command must be called before all others. This command will assure, that client and server speak the same version of the protocol.
\item {\it Parmameters:}
\begin{itemize}
\item protocol\_version
\item client\_version
\item client\_uniqid
\end{itemize}
\item {\it Result:}\\
The reply contains 4 cells: protocol version, server version, client id for further requests, ip the connection came from
\end{description}

\subsection{req\_full}

\begin{description}
\item {\it Description:}\\
Returns a list of races and the drivers in this races.
\item {\it Parmameters:}
\begin{itemize}
\item client\_id
\end{itemize}
\item {\it Result:}\\
The complete current racelist. Each line holds either a race or following the drivers of a race. Each line starts either with a cell containing R or D. Races consist of the following: 
				
			R, 
			server\_id, 
			ip, 
			joinport, 
			name, 
			info1, 
			info2, 
			comment, 
			isdedicatedserver, 
			ispassworded, 
			isbosspassworded, 
			isauthenticedserver, 
			allowedchassis, 
			allowedcarclasses, 
			allowsengineswapping, 
			modindent, 
			maxlatency, 
			bandwidth, 
			players,
			maxplayers, 
			trackdir, 
			racetype, 
			praclength, 
			sessionleft, 
			sessiontype,
			aiplayers,
			numraces,
			repeatcount,
			flags


			The drivers contain this data:

			D,
			firstname,
			lastname,
			class\_id,
			team\_id,
			mod\_id,
			nationality,
			helmet\_colour,
			qualifying\_time,
			race\_position,
			race\_laps,
			race\_notes
			
\end{description}

\subsection{host}

\begin{description}
\item {\it Description:}\\
Starts hosting of a race. The given informations are used to describe the race and will be displayed in the same order in the racelist.
\item {\it Parmameters:}
\begin{itemize}
\item client\_id
\item joinport
\item name
\item info1
\item info2
\item comment
\item isdedicatedserver
\item ispassworded
\item isbosspassworded
\item isauthenticedserver
\item allowedchassis
\item allowedcarclasses
\item allowsengineswapping
\item modindent
\item maxlatency
\item bandwidth
\item maxplayers
\item trackdir
\item racetype
\item praclength
\item aiplayers
\item numraces
\item repeatcount
\item flags
\item firstname
\item lastname
\item class\_id
\item team\_id
\item mod\_id
\item nationality
\item helmet\_colour
\end{itemize}
\item {\it Result:}\\
A unique id for the server, that will be used to update the hosting and race informations and also by the clients to join/leave the race.
\end{description}



\end{document}
